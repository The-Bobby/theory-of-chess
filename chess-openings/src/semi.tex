\begin{minipage}[t]{.175\linewidth}
\fenboard{rnbqkb1r/pp2pppp/3p1n2/8/3NP3/2N5/PPP2PPP/R1BQKB1R b KQkq - 2 5}
\raggedright
\begin{center}
\scalebox{.560}{\showboard}
\end{center}
\newgame
%FEN@rnbqkb1r/pp2pppp/3p1n2/8/3NP3/2N5/PPP2PPP/R1BQKB1R b KQkq - 2 5
\emph{Sicilian defence}: \mainline{1. e4 c5}, usually followed by \mainline{2. Nf3}.
After \mainline{2...d6 3. d4 cxd4 4. Nxd4 Nf6 5. Nc3} (diagram) Black can choose between four major variations:
\emph{Najdorf} \variation{5...a6},
\emph{Dragon} \variation{5...g6},
\emph{Classical} \variation{5...Nc6} or
\emph{Scheveningen} \variation{5...e6}.
\vspace{2mm}
\end{minipage}
\hspace{5mm}
\begin{minipage}[t]{.175\linewidth}
\fenboard{r1bqkbnr/pp1ppppp/2n5/8/3NP3/8/PPP2PPP/RNBQKB1R b KQkq - 0 4}
\raggedright
\begin{center}
\scalebox{.560}{\showboard}
\end{center}
\newgame
%FEN@r1bqkbnr/pp1ppppp/2n5/8/3NP3/8/PPP2PPP/RNBQKB1R b KQkq - 0 4
\emph{Sicilian defence}: \mainline{1. e4 c5}.
After \mainline{2. Nf3 Nc6 3. d4 cxd4 4. Nxd4} (diagram) common move is \mainline{4...Nf6}.
Other important moves are:
\emph{transposed Taimanov Variation} \variation{4...e6},
\emph{Accelerated Dragon} \variation{4...g6} and
\emph{Kalashnikov Variation} \variation{4...e5}.
\vspace{2mm}
\end{minipage}
\hspace{5mm}
\begin{minipage}[t]{.175\linewidth}
\fenboard{rnbqkbnr/pp1p1ppp/4p3/8/3NP3/8/PPP2PPP/RNBQKB1R b KQkq - 0 4}
\raggedright
\begin{center}
\scalebox{.560}{\showboard}
\end{center}
\newgame
%FEN@rnbqkbnr/pp1p1ppp/4p3/8/3NP3/8/PPP2PPP/RNBQKB1R b KQkq - 0 4
\emph{Sicilian defence}: \mainline{1. e4 c5}.
After \mainline{2. Nf3 e6 3. d4 cxd4 4. Nxd4} (diagram) Black has three main moves:
\emph{Taimanov Variation} \variation{4...Nc6},
\emph{Kan Variation} \variation{4...a6} and 
\variation{4...Nf6}.
\vspace{2mm}
\end{minipage}
\hspace{5mm}
\begin{minipage}[t]{.175\linewidth}
\fenboard{r1bq1rk1/pp2ppbp/2np1np1/8/2BNP3/2N1BP2/PPPQ2PP/R3K2R b KQ - 4 9}
\raggedright
\begin{center}
\scalebox{.560}{\showboard}
\end{center}
\newgame
%FEN@r1bq1rk1/pp2ppbp/2np1np1/8/2BNP3/2N1BP2/PPPQ2PP/R3K2R b KQ - 4 9
\emph{Sicilian Defence}: \mainline{1. e4 c5}.
\emph{Dragon Variation}: \mainline{2. Nf3 d6 3. d4 cxd4 4. Nxd4 Nf6 5. Nc3 g6}.
\emph{Yugoslav Attack}: \mainline{6. Be3 Bg7 7. f3 O-O 8. Qd2 Nc6 9. Bc4}.
Main line: \mainline{9... Bd7 10. 0-0-0 Rc8 11. Bb3 Ne5 12. Kb1 Re8} (diagram).

\vspace{2mm}
\end{minipage}
\hspace{5mm}
\begin{minipage}[t]{.175\linewidth}
\fenboard{rnbqk1nr/pp3ppp/4p3/2ppP3/3P4/P1P5/2P2PPP/R1BQKBNR b KQkq - 0 6}
\raggedright
\begin{center}
\scalebox{.560}{\showboard}
\end{center}
\newgame
%FEN@rnbqk1nr/pp3ppp/4p3/2ppP3/3P4/P1P5/2P2PPP/R1BQKBNR b KQkq - 0 6
\emph{French defence}: \mainline{1. e4 e6}, usually followed by \mainline{2. d4 d5 3. Nc3}.
Black has three options:
\emph{Winawer} \mainline{3...Bb4}, main line then is: \mainline{4. e5 c5 5. a3 Bxc3+ 6. bxc3} (diagram);
\emph{Classical} \variation{3...Nf6};
\emph{Rubinstein} \variation{3...dxe4}.\vspace{2mm}
\end{minipage}
\newline
\begin{minipage}[t]{.175\linewidth}
\fenboard{rnbqk2r/ppp2ppp/4pb2/8/3PN3/5N2/PPP2PPP/R2QKB1R b KQkq - 1 7}
\raggedright
\begin{center}
\scalebox{.560}{\showboard}
\end{center}
\newgame
%FEN@rnbqk2r/ppp2ppp/4pb2/8/3PN3/5N2/PPP2PPP/R2QKB1R b KQkq - 1 7
\emph{French defence}: \mainline{1. e4 e6 2. d4 d5 3. Nc3}.
\emph{Burn variation} is a line in classical variation: \mainline{3...Nf6 4. Bg5 dxe4 5. Nxe4}, usually followed by \mainline{5...Be7 6. Bxf6 Bxf6 7. Nf3} (diagram).\vspace{2mm}
\end{minipage}
\hspace{5mm}
\begin{minipage}[t]{.175\linewidth}
\fenboard{rnbqkbnr/pp2pppp/2p5/3p4/3PP3/2N5/PPP2PPP/R1BQKBNR b KQkq - 1 3}
\raggedright
\begin{center}
\scalebox{.560}{\showboard}
\end{center}
\newgame
%FEN@rnbqkbnr/pp2pppp/2p5/3p4/3PP3/2N5/PPP2PPP/R1BQKBNR b KQkq - 1 3
\emph{Caro-Kann defence}: \mainline{1. e4 c6}, after \mainline{2. d4 d5} common moves are \mainline{3. Nc3} (diagram), \variation{3. Nd2}, \variation{3. exd5} and \variation{3. e5}.\vspace{2mm}
\end{minipage}
\hspace{5mm}
\begin{minipage}[t]{.175\linewidth}
\fenboard{rnbq1rk1/ppp1ppbp/3p1np1/8/3PPP2/2N2N2/PPP3PP/R1BQKB1R w KQ - 3 6}
\raggedright
\begin{center}
\scalebox{.560}{\showboard}
\end{center}
\newgame
%FEN@rnbq1rk1/ppp1ppbp/3p1np1/8/3PPP2/2N2N2/PPP3PP/R1BQKB1R w KQ - 3 6
\emph{Pirc defence}: \mainline{1. e4 d6 2. d4 Nf6 3. Nc3 g6} followed by \mainline{4. f4 Bg7 5. Nf3 O-O} (Austrian attack, diagram) or \variation{4. Nf3 Bg7 5. Be2 O-O 6. O-O} (two knights system),\vspace{2mm}
\end{minipage}
\hspace{5mm}
\begin{minipage}[t]{.175\linewidth}
\fenboard{rnb1kbnr/ppp1pppp/8/q7/8/2N5/PPPP1PPP/R1BQKBNR w KQkq - 2 4}
\raggedright
\begin{center}
\scalebox{.560}{\showboard}
\end{center}
\newgame
%FEN@rnb1kbnr/ppp1pppp/8/q7/8/2N5/PPPP1PPP/R1BQKBNR w KQkq - 2 4
\emph{Scandinavian defence}: \mainline{1. e4 d5 2. exd5} followed by \mainline{2...Qxd5 3. Nc3} and then \mainline{3...Qa5} (diagram) or \variation{3...Qd8} or \variation{3...Qd6}. 
Another main branch is \variation{2...Nf6 3. d4 Nxd5 4. c4},\vspace{2mm}
\end{minipage}
\hspace{5mm}
\begin{minipage}[t]{.175\linewidth}
\fenboard{rn1qk1nr/pp2ppbp/2pp2p1/8/3PPPb1/2N2N2/PPP3PP/R1BQKB1R w KQkq - 2 6}
\raggedright
\begin{center}
\scalebox{.560}{\showboard}
\end{center}
\newgame
%FEN@rn1qk1nr/pp2ppbp/2pp2p1/8/3PPPb1/2N2N2/PPP3PP/R1BQKB1R w KQkq - 2 6
\emph{Modern defence}: \mainline{1. e4 g6} followed by \mainline{2. d4 Bg7 3. Nc3 d6 4. f4 c6 5. Nf3 Bg4}.

\vspace{2mm}
\end{minipage}
\newline
