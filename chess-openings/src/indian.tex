\begin{minipage}[t]{.225\linewidth}
\fenboard{rnbqkb1r/pp3p1p/3p1np1/2pP4/8/2N2N2/PP2PPPP/R1BQKB1R w KQkq - 0 7}
\raggedright
\begin{center}
\scalebox{.700}{\showboard}
\end{center}
\newgame
%FEN@rnbqkb1r/pp3p1p/3p1np1/2pP4/8/2N2N2/PP2PPPP/R1BQKB1R w KQkq - 0 7
\emph{Modern Benoni}: \mainline{1. d4 Nf6 2. c4 c5 3. d5 e6}, can be followed by \mainline{4. Nc3 exd5 5. cxd5 d6 6. Nf3 g6} and then classical: \mainline{7. Nf3 Bg7 8. Be2 O-O 9. O-O} (diagram) or modern main line: \variation{7. Nf3 Bg7 8. h3 O-O 9. Bd3}\vspace{2mm}
\end{minipage}
\hspace{5mm}
\begin{minipage}[t]{.225\linewidth}
\fenboard{rnbqkb1r/pp3p1p/3p1np1/2pP4/4P3/2N5/PP3PPP/R1BQKBNR w KQkq - 0 7}
\raggedright
\begin{center}
\scalebox{.700}{\showboard}
\end{center}
\newgame
%FEN@rnbqkb1r/pp3p1p/3p1np1/2pP4/4P3/2N5/PP3PPP/R1BQKBNR w KQkq - 0 7
\emph{Modern Benoni}: \mainline{1. d4 Nf6 2. c4 c5 3. d5 e6}, can be followed by \mainline{4. Nc3 exd5 5. cxd5 d6 6. e4 g6} (diagram).\vspace{2mm}
\end{minipage}
\hspace{5mm}
\begin{minipage}[t]{.225\linewidth}
\fenboard{8/8/8/8/8/8/8/8 w - - 0 1}
\raggedright
\begin{center}
\scalebox{.700}{\showboard}
\end{center}
\newgame
%FEN@8/8/8/8/8/8/8/8 w - - 0 1
\emph{Benko/Volga Gambit}
\vspace{2mm}
\end{minipage}
\hspace{5mm}
\begin{minipage}[t]{.225\linewidth}
\fenboard{rnbq1rk1/pp3ppp/4pn2/2pp4/1bPP4/2NBPN2/PP3PPP/R1BQ1RK1 b - - 1 7}
\raggedright
\begin{center}
\scalebox{.700}{\showboard}
\end{center}
\newgame
%FEN@rnbq1rk1/pp3ppp/4pn2/2pp4/1bPP4/2NBPN2/PP3PPP/R1BQ1RK1 b - - 1 7
\emph{Nimzo-Indian Defence}: \mainline{1.d4 Nf6 2.c4 e6 3.Nc3 Bb4}, common reply is Rubinstein system: \mainline{4. e3} with main line: \mainline{4...O-O 5. Bd3 d5 6. Nf3 c5 7. O-O} (diagram).\vspace{2mm}
\end{minipage}
\newline
\begin{minipage}[t]{.225\linewidth}
\fenboard{8/8/8/8/8/8/8/8 w - - 0 1}
\raggedright
\begin{center}
\scalebox{.700}{\showboard}
\end{center}
\newgame
%FEN@8/8/8/8/8/8/8/8 w - - 0 1
\emph{Queen's Indian Defence}
\vspace{2mm}
\end{minipage}
\hspace{5mm}
\begin{minipage}[t]{.225\linewidth}
\fenboard{rnbqk2r/ppp1bppp/4pn2/8/2pP4/5NP1/PP2PPBP/RNBQK2R w KQkq - 2 6}
\raggedright
\begin{center}
\scalebox{.700}{\showboard}
\end{center}
\newgame
%FEN@rnbqk2r/ppp1bppp/4pn2/8/2pP4/5NP1/PP2PPBP/RNBQK2R w KQkq - 2 6
\emph{Catalan Opening}: \mainline{1. d4 Nf6 2. c4 e6 3. g3} with classical line: \mainline{3...d5 4.Bg2 dxc4 5.Nf3 Be7} (diagram).\vspace{2mm}
\end{minipage}
\hspace{5mm}
\begin{minipage}[t]{.225\linewidth}
\fenboard{rnbqkb1r/ppp1pp1p/6p1/8/3PP3/2P5/P4PPP/R1BQKBNR b KQkq - 0 6}
\raggedright
\begin{center}
\scalebox{.700}{\showboard}
\end{center}
\newgame
%FEN@rnbqkb1r/ppp1pp1p/6p1/8/3PP3/2P5/P4PPP/R1BQKBNR b KQkq - 0 6
\emph{Grünfeld Defence}: \mainline{1. d4 Nf6 2. c4 g6 3. Nc3 d5}.
Exchange variation: \mainline{4. cxd5 Nxd5 5. e4 Nxc3 6. bxc3} (diagram).
\vspace{2mm}
\end{minipage}
\hspace{5mm}
\begin{minipage}[t]{.225\linewidth}
\fenboard{rnbq1rk1/ppp1ppbp/5np1/8/2QPP3/2N2N2/PP3PPP/R1B1KB1R b KQ e3 0 7}
\raggedright
\begin{center}
\scalebox{.700}{\showboard}
\end{center}
\newgame
%FEN@rnbq1rk1/ppp1ppbp/5np1/8/2QPP3/2N2N2/PP3PPP/R1B1KB1R b KQ e3 0 7
\emph{Grünfeld Defence}: \mainline{1. d4 Nf6 2. c4 g6 3. Nc3 d5}.
Russian system: \mainline{4.Nf3 Bg7 5.Qb3 dxc4 6. Qxc4 0-0 7. e4} (diagram).
Another variation is Taimanov's \variation{4.Nf3 Bg7 5. Bg5}\vspace{2mm}
\end{minipage}
\newline
\begin{minipage}[t]{.225\linewidth}
\fenboard{rnbq1rk1/ppp2pbp/3p1np1/4p3/2PPP3/2N2N2/PP2BPPP/R1BQK2R w KQ e6 0 7}
\raggedright
\begin{center}
\scalebox{.700}{\showboard}
\end{center}
\newgame
%FEN@rnbq1rk1/ppp2pbp/3p1np1/4p3/2PPP3/2N2N2/PP2BPPP/R1BQK2R w KQ e6 0 7
\emph{King's Indian Defence}: \mainline{1. d4 Nf6 2. c4 g6}, classical variation: \mainline{3. Nc3 Bg7 4. e4 d6 5. Nf3 O-O 6. Be2 e5} (diagram).\vspace{2mm}
\end{minipage}
\hspace{5mm}
\begin{minipage}[t]{.225\linewidth}
\fenboard{rnbq1rk1/ppp1ppbp/3p1np1/8/2PP4/5NP1/PP2PPBP/RNBQ1RK1 b - - 1 6}
\raggedright
\begin{center}
\scalebox{.700}{\showboard}
\end{center}
\newgame
%FEN@rnbq1rk1/ppp1ppbp/3p1np1/8/2PP4/5NP1/PP2PPBP/RNBQ1RK1 b - - 1 6
\emph{King's Indian Defence}: \mainline{1. d4 Nf6 2. c4 g6}, fianchetto variation: \mainline{3. Nf3 Bg7 4. g3 O-O 5. Bg2 d6 6. O-O} (diagram).\vspace{2mm}
\end{minipage}
\hspace{5mm}

