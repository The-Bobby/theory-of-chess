\begin{minipage}[t]{.175\linewidth}
\fenboard{r1bq1rk1/2p1bppp/p1np1n2/1p2p3/4P3/1BP2N2/PP1P1PPP/RNBQR1K1 w - - 1 9}
\raggedright
\begin{center}
\scalebox{.560}{\showboard}
\end{center}
\newgame
%FEN@r1bq1rk1/2p1bppp/p1np1n2/1p2p3/4P3/1BP2N2/PP1P1PPP/RNBQR1K1 w - - 1 9
\openingname{Ruy Lopez}: \mainline{1. e4 e5 2. Nf3 Nc6 3. Bb5}.
\openingname{Morphy defence}: \mainline{3...a6}.
Main line: \mainline{4. Ba4 Nf6 5. O-O Be7 6. Re1 b5 7. Bb3 d6 8. c3 O-O} (diagram).
\vspace{2mm}
\end{minipage}
\hspace{5mm}
\begin{minipage}[t]{.175\linewidth}
\fenboard{r1bk1b1r/ppp2ppp/2p5/4Pn2/8/5N2/PPP2PPP/RNB2RK1 w - - 0 9}
\raggedright
\begin{center}
\scalebox{.560}{\showboard}
\end{center}
\newgame
%FEN@r1bk1b1r/ppp2ppp/2p5/4Pn2/8/5N2/PPP2PPP/RNB2RK1 w - - 0 9
\openingname{Ruy Lopez}: \mainline{1. e4 e5 2. Nf3 Nc6 3. Bb5}.
\openingname{Berlin Defence}: \mainline{3...Nf6}.
Main line: \mainline{4. O-O Nxe4 5. d4 Nd6 6. Bxc6 dxc6 7. dxe5 Nf5 8. Qxd8+ Kxd8} (diagram).
\vspace{2mm}
\end{minipage}
\hspace{5mm}
\begin{minipage}[t]{.175\linewidth}
\fenboard{r1bqk2r/pppp1ppp/2n2n2/2b1p3/2B1P3/2PP1N2/PP3PPP/RNBQK2R b KQkq - 0 5}
\raggedright
\begin{center}
\scalebox{.560}{\showboard}
\end{center}
\newgame
%FEN@r1bqk2r/pppp1ppp/2n2n2/2b1p3/2B1P3/2PP1N2/PP3PPP/RNBQK2R b KQkq - 0 5
\openingname{Italian game}: \mainline{1. e4 e5 2. Nf3 Nc6 3. Bc4}.
\openingname{Giuoco Piano}: \mainline{3...Bc5}.
\openingname{Giuoco Pianissimo}: \mainline{4. c3 Nf6 5. d3} (diagram) or \openingname{Evans Gambit}: \variation{4. b4} followed by \variation{4...Bxb4 5. c3}.
\vspace{2mm}
\end{minipage}
\hspace{5mm}
\begin{minipage}[t]{.175\linewidth}
\fenboard{r1bqkb1r/ppp2ppp/5n2/n2Pp1N1/2B5/8/PPPP1PPP/RNBQK2R w KQkq - 1 6}
\raggedright
\begin{center}
\scalebox{.560}{\showboard}
\end{center}
\newgame
%FEN@r1bqkb1r/ppp2ppp/5n2/n2Pp1N1/2B5/8/PPPP1PPP/RNBQK2R w KQkq - 1 6
\openingname{Italian game}: \mainline{1. e4 e5 2. Nf3 Nc6 3. Bc4}.
\openingname{Two Knights Defence}: \mainline{3...Nf6}.
Main line: \mainline{4. Ng5 d5 5. exd5 Na5} (diagram).
\openingname{Modern Bishop’s Opening}: \variation{4. d3} followed by \variation{4...Be7 5. O-O O-O 6. Re1 d6}.
\vspace{2mm}
\end{minipage}
\hspace{5mm}
\begin{minipage}[t]{.175\linewidth}
\fenboard{r1bqkbnr/pppp1ppp/2n5/8/3NP3/8/PPP2PPP/RNBQKB1R b KQkq - 0 4}
\raggedright
\begin{center}
\scalebox{.560}{\showboard}
\end{center}
\newgame
%FEN@r1bqkbnr/pppp1ppp/2n5/8/3NP3/8/PPP2PPP/RNBQKB1R b KQkq - 0 4
\openingname{Scotch game}: \mainline{1. e4 e5 2. Nf3 Nc6 3. d4 exd4}.
Main line: \mainline{4. Nxd4} (diagram), followed by \openingname{Classical}: \variation{4...Bc5} or \openingname{Schmidt variation}: \variation{4...Nf6}.
Other lines: \openingname{Scotch Gambit} \variation{4. Bc4}, \openingname{Göring Gambit} \variation{4. c3}.
\vspace{2mm}
\end{minipage}
\newline
\begin{minipage}[t]{.175\linewidth}
\fenboard{r1bqk2r/pppp1ppp/2n2n2/1B2p3/1b2P3/2N2N2/PPPP1PPP/R1BQK2R w KQkq - 6 5}
\raggedright
\begin{center}
\scalebox{.560}{\showboard}
\end{center}
\newgame
%FEN@r1bqk2r/pppp1ppp/2n2n2/1B2p3/1b2P3/2N2N2/PPPP1PPP/R1BQK2R w KQkq - 6 5
\openingname{Four Knights Game}: \mainline{1. e4 e5 2. Nf3 Nc6 3. Nc3 Nf6}.
Spanish lines: \openingname{Double Spanish} \mainline{4. Bb5 Bb4} (diagram), \openingname{Rubinstein} \variation{4...Nd4} or \openingname{Classical variation} \variation{4...Bc5}.
\vspace{2mm}
\end{minipage}
\hspace{5mm}
\begin{minipage}[t]{.175\linewidth}
\fenboard{r1bqk2r/p1pp1ppp/2p2n2/8/1b2P3/2NB4/PPP2PPP/R1BQK2R b KQkq - 1 7}
\raggedright
\begin{center}
\scalebox{.560}{\showboard}
\end{center}
\newgame
%FEN@r1bqk2r/p1pp1ppp/2p2n2/8/1b2P3/2NB4/PPP2PPP/R1BQK2R b KQkq - 1 7
\openingname{Four Knights Game}: \mainline{1. e4 e5 2. Nf3 Nc6 3. Nc3 Nf6}.
\openingname{Scotch Four Knights Game}: \mainline{4. d4}, main line: \mainline{4...exd4 5. Nxd4 Bb4 6. Nxc6 bxc6 7. Bd3} (diagram).
\vspace{2mm}
\end{minipage}
\hspace{5mm}
\begin{minipage}[t]{.175\linewidth}
\fenboard{rnbqkb1r/ppp2ppp/3p4/8/4n3/5N2/PPPP1PPP/RNBQKB1R w KQkq - 0 5}
\raggedright
\begin{center}
\scalebox{.560}{\showboard}
\end{center}
\newgame
%FEN@rnbqkb1r/ppp2ppp/3p4/8/4n3/5N2/PPPP1PPP/RNBQKB1R w KQkq - 0 5
\openingname{Petrov's Defence}: \mainline{1. e4 e5 2. Nf3 Nf6}.
\openingname{Classical variation}: \mainline{3. Nxe5} followed by \mainline{3...d6 4. Nf3 Nxe4} (diagram).
\openingname{Steinitz attack}: \variation{3. d4} followed by \variation{3...Nxe4} or \variation{3...exd4 4. e5 Ne4}.
\vspace{2mm}
\end{minipage}
\hspace{5mm}
\begin{minipage}[t]{.175\linewidth}
\fenboard{rnbqkbnr/ppp3pp/3p4/4pp2/3PP3/5N2/PPP2PPP/RNBQKB1R w KQkq f6 0 4}
\raggedright
\begin{center}
\scalebox{.560}{\showboard}
\end{center}
\newgame
%FEN@rnbqkbnr/ppp3pp/3p4/4pp2/3PP3/5N2/PPP2PPP/RNBQKB1R w KQkq f6 0 4
\openingname{Philidor Defence}: \mainline{1. e4 e5 2. Nf3 d6}.
Main line: \mainline{3. d4} followed by \openingname{Philidor counter gambit} \mainline{3...f5} (diagram).
Other lines: \openingname{exchange} \variation {3...exd4}, \openingname{Nimzowitch} \variation {3...Nf6} or \openingname{Hanham variation} \variation {3...Nd7}.
\vspace{2mm}
\end{minipage}
\hspace{5mm}
\begin{minipage}[t]{.175\linewidth}
\fenboard{rnbqkbnr/pppp1p1p/8/6p1/4Pp1P/5N2/PPPP2P1/RNBQKB1R b KQkq h3 0 4}
\raggedright
\begin{center}
\scalebox{.560}{\showboard}
\end{center}
\newgame
%FEN@rnbqkbnr/pppp1p1p/8/6p1/4Pp1P/5N2/PPPP2P1/RNBQKB1R b KQkq h3 0 4
\openingname{King's Gambit accepted}: \mainline{1. e4 e5 2. f4 exf4}.
\openingname{Paris attack}: \mainline{3. Nf3 g5 4. h4} (diagram).
Other lines: \variation{4. Bc4}, \variation{4. Nc3}.
\openingname{Modern defence}: \variation{3...d5}.
\openingname{Bishop's Gambit}: \variation{3. Bc4} followed by \variation{3...Nf6} or \variation{3...d5}.
\vspace{2mm}
\end{minipage}
\newline
\begin{minipage}[t]{.175\linewidth}
\fenboard{rnbqkbnr/ppp2ppp/8/3Pp3/5P2/8/PPPP2PP/RNBQKBNR b KQkq - 0 3}
\raggedright
\begin{center}
\scalebox{.560}{\showboard}
\end{center}
\newgame
%FEN@rnbqkbnr/ppp2ppp/8/3Pp3/5P2/8/PPPP2PP/RNBQKBNR b KQkq - 0 3
\openingname{King's Gambit declined}: \mainline{1. e4 e5 2. f4}.
\openingname{Falkbeer countergambit}: \mainline{2...d5 3. exd5} (diagram) followed by \mainline{3...e4} or \variation{3...c6}.
\openingname{Classical variation}: \variation{2...Bc5 3. Nf3 d6}.
\vspace{2mm}
\end{minipage}
\hspace{5mm}
\begin{minipage}[t]{.175\linewidth}
\fenboard{r1bqkbnr/pppp1ppp/2n5/8/4P3/4Q3/PPP2PPP/RNB1KBNR b KQkq - 2 4}
\raggedright
\begin{center}
\scalebox{.560}{\showboard}
\end{center}
\newgame
%FEN@r1bqkbnr/pppp1ppp/2n5/8/4P3/4Q3/PPP2PPP/RNB1KBNR b KQkq - 2 4
\openingname{Center Game}: \mainline{1. e4 e5 2. d4 exd4}.
Universal sequence is \mainline{3. Qxd4 Nc6} and \openingname{Paulsen's attack}: \mainline{4. Qe3} (diagram)
\vspace{2mm}
\end{minipage}
\hspace{5mm}
\begin{minipage}[t]{.175\linewidth}
\fenboard{rnbqkbnr/pppp1ppp/8/8/4P3/2N5/PP3PPP/R1BQKBNR b KQkq - 0 4}
\raggedright
\begin{center}
\scalebox{.560}{\showboard}
\end{center}
\newgame
%FEN@rnbqkbnr/pppp1ppp/8/8/4P3/2N5/PP3PPP/R1BQKBNR b KQkq - 0 4
\openingname{Danish gambit accepted}: \mainline{1. e4 e5 2. d4 exd4 3. c3 dxc3}.
\openingname{Alekhine variation}: \mainline{4. Nxc3} (diagram).
\openingname{Lindehn's continuation} \variation{4. Bc4}.
\variation{3...d6}, \variation{3...Qe7} or \variation{3...d5} to decline.
\vspace{2mm}
\end{minipage}
\hspace{5mm}

