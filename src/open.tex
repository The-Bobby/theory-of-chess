\begin{figure}[H]
\begin{minipage}[b]{.196\linewidth}
\begin{framed}
\raggedright
\newgame
\emph{Ruy Lopez}: \mainline{1. e4 e5 2. Nf3 Nc6 3. Bb5}.
\emph{Morphy defence}: \mainline{3...a6}.
Main line: \mainline{4. Ba4 Nf6 5. O-O Be7 6. Re1 b5 7. Bb3 d6 8. c3 O-O} (diagram).
\begin{center}
\scalebox{.560}{\showboard}
\end{center}
% \subcaption{...}
\end{framed}
\end{minipage}
\begin{minipage}[b]{.196\linewidth}
\begin{framed}
\raggedright
\newgame
\emph{Ruy Lopez}: \mainline{1. e4 e5 2. Nf3 Nc6 3. Bb5}.
\emph{Berlin Defence}: \mainline{3...Nf6}.
Main line: \mainline{4. O-O Nxe4 5. d4 Nd6 6. Bxc6 dxc6 7. dxe5 Nf5 8. Qxd8+ Kxd8} (diagram).
\begin{center}
\scalebox{.560}{\showboard}
\end{center}
% \subcaption{...}
\end{framed}
\end{minipage}
\begin{minipage}[b]{.196\linewidth}
\begin{framed}
\raggedright
\newgame
\emph{Italian game}: \mainline{1. e4 e5 2. Nf3 Nc6 3. Bc4}.
\emph{Giuoco Piano}: \mainline{3...Bc5}.
\emph{Giuoco Pianissimo}: \mainline{4. c3 Nf6 5. d3} (diagram) or \emph{Evans Gambit}: \variation{4. b4} followed by \variation{4...Bxb4 5. c3}.
\begin{center}
\scalebox{.560}{\showboard}
\end{center}
% \subcaption{...}
\end{framed}
\end{minipage}
\begin{minipage}[b]{.196\linewidth}
\begin{framed}
\raggedright
\newgame
\emph{Italian game}: \mainline{1. e4 e5 2. Nf3 Nc6 3. Bc4}.
\emph{Two Knights Defence}: \mainline{3...Nf6}.
Main line: \mainline{4. Ng5 d5 5. exd5 Na5} (diagram), other lines: \emph{Modern Bishop’s Opening} \variation{4. d3} followed by \variation{4...Be7 5. O-O O-O 6. Re1 d6}.\begin{center}
\scalebox{.560}{\showboard}
\end{center}
% \subcaption{...}
\end{framed}
\end{minipage}
\begin{minipage}[b]{.196\linewidth}
\begin{framed}
\raggedright
\newgame
\emph{Scotch game}: \mainline{1.e4 e5 2.Nf3 Nc6 3.d4 exd4}.
Main line: \mainline{4. Nxd4} (diagram), followed by \emph{Classical}: \variation{4...Bc5} or \emph{Schmidt variation}: \variation{4...Nf6}.
Other lines: \emph{Scotch Gambit} \variation{4. Bc4}, \emph{Göring Gambit} \variation{4. c3}.
\begin{center}
\scalebox{.560}{\showboard}
\end{center}
% \subcaption{...}
\end{framed}
\end{minipage}
\end{figure}

\begin{figure}[H]
\begin{minipage}[b]{.196\linewidth}
\begin{framed}
\raggedright
\newgame
\emph{Four Knights Game}: \mainline{1. e4 e5 2. Nf3 Nc6 3. Nc3 Nf6}.
Spanish lines: \emph{Double Spanish} \mainline{4. Bb5 Bb4} (diagram), \emph{Rubinstein} \variation{4...Nd4} or \emph{Classical variation} \variation{4...Bc5}.
\begin{center}
\scalebox{.560}{\showboard}
\end{center}
% \subcaption{...}
\end{framed}
\end{minipage}
\begin{minipage}[b]{.196\linewidth}
\begin{framed}
\raggedright
\newgame
\emph{Four Knights Game}: \mainline{1. e4 e5 2. Nf3 Nc6 3. Nc3 Nf6}.
\emph{Scotch Four Knights Game}: \mainline{4. d4}, main line: \mainline{4...exd4 5. Nxd4 Bb4 6. Nxc6 bxc6 7. Bd3} (diagram).
\begin{center}
\scalebox{.560}{\showboard}
\end{center}
% \subcaption{...}
\end{framed}
\end{minipage}
\begin{minipage}[b]{.196\linewidth}
\begin{framed}
\raggedright
\newgame
\emph{Petrov's Defence}: \mainline{1. e4 e5 2. Nf3 Nf6}.
\emph{Classical variation}: \mainline{3. Nxe5} followed by \mainline{3...d6 4. Nf3 Nxe4} (diagram).
\emph{Steinitz attack}: \variation{3. d4} followed by \variation{3...Nxe4} or \variation{3...exd4 4. e5 Ne4}.\begin{center}
\scalebox{.560}{\showboard}
\end{center}
% \subcaption{...}
\end{framed}
\end{minipage}
\begin{minipage}[b]{.196\linewidth}
\begin{framed}
\raggedright
\newgame
\emph{Philidor Defence}: \mainline{1. e4 e5 2. Nf3 d6}.
Main line: \mainline{3. d4} followed by \emph{Philidor counter gambit} \mainline{3...f5} (diagram).
Other lines: \emph{exchange} \variation {3...exd4}, \emph{Nimzowitch} \variation {3...Nf6} or \emph{Hanham variation} \variation {3...Nd7}.\begin{center}
\scalebox{.560}{\showboard}
\end{center}
% \subcaption{...}
\end{framed}
\end{minipage}
\begin{minipage}[b]{.196\linewidth}
\begin{framed}
\raggedright
\newgame
\emph{King's Gambit accepted}: \mainline{1. e4 e5 2. f4 exf4}.
\emph{Paris attack}: \mainline{3. Nf3 g5 4. h4} (diagram).
Other lines: \variation{4. Bc4}, \variation{4. Nc3}.
\emph{Modern defence}: \variation{3...d5}.
\emph{Bishop's Gambit}: \variation{3. Bc4} followed by \variation{3...Nf6} or \variation{3...d5}.
\begin{center}
\scalebox{.560}{\showboard}
\end{center}
% \subcaption{...}
\end{framed}
\end{minipage}
\end{figure}

\begin{figure}[H]
\begin{minipage}[b]{.196\linewidth}
\begin{framed}
\raggedright
\newgame
\emph{King's Gambit declined}: \mainline{1. e4 e5 2. f4}.
\emph{Falkbeer countergambit}: \mainline{2...d5 3. exd5} (diagram) followed by \mainline{3...e4} or \variation{3...c6}.
\emph{Classical variation}: \variation{2...Bc5 3. Nf3 d6}.
\begin{center}
\scalebox{.560}{\showboard}
\end{center}
% \subcaption{...}
\end{framed}
\end{minipage}
\begin{minipage}[b]{.196\linewidth}
\begin{framed}
\raggedright
\newgame
\emph{Center Game}: \mainline{1. e4 e5 2. d4 exd4}.
Universal sequence is \mainline{3. Qxd4 Nc6} and \emph{Paulsen's attack}: \mainline{4. Qe3} (diagram)
\begin{center}
\scalebox{.560}{\showboard}
\end{center}
% \subcaption{...}
\end{framed}
\end{minipage}
\begin{minipage}[b]{.196\linewidth}
\begin{framed}
\raggedright
\newgame
\emph{Danish gambit accepted}: \mainline{1. e4 e5 2. d4 exd4 3. c3 dxc3},.
\emph{Alekhine variation}: \mainline{4. Nxc3} (diagram).
\emph{Lindehn's continuation} \variation{4. Bc4}.
\variation{3...d6}, \variation{3...Qe7} or \variation{3...d5} to decline.
\begin{center}
\scalebox{.560}{\showboard}
\end{center}
% \subcaption{...}
\end{framed}
\end{minipage}
\end{figure}

